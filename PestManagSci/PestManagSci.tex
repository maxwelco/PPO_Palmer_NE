% Options for packages loaded elsewhere
\PassOptionsToPackage{unicode}{hyperref}
\PassOptionsToPackage{hyphens}{url}
%
\documentclass[
  12pt,
  a4paper]{article}
\usepackage{lmodern}
\usepackage{amssymb,amsmath}
\usepackage{ifxetex,ifluatex}
\ifnum 0\ifxetex 1\fi\ifluatex 1\fi=0 % if pdftex
  \usepackage[T1]{fontenc}
  \usepackage[utf8]{inputenc}
  \usepackage{textcomp} % provide euro and other symbols
\else % if luatex or xetex
  \usepackage{unicode-math}
  \defaultfontfeatures{Scale=MatchLowercase}
  \defaultfontfeatures[\rmfamily]{Ligatures=TeX,Scale=1}
\fi
% Use upquote if available, for straight quotes in verbatim environments
\IfFileExists{upquote.sty}{\usepackage{upquote}}{}
\IfFileExists{microtype.sty}{% use microtype if available
  \usepackage[]{microtype}
  \UseMicrotypeSet[protrusion]{basicmath} % disable protrusion for tt fonts
}{}
\makeatletter
\@ifundefined{KOMAClassName}{% if non-KOMA class
  \IfFileExists{parskip.sty}{%
    \usepackage{parskip}
  }{% else
    \setlength{\parindent}{0pt}
    \setlength{\parskip}{6pt plus 2pt minus 1pt}}
}{% if KOMA class
  \KOMAoptions{parskip=half}}
\makeatother
\usepackage{xcolor}
\IfFileExists{xurl.sty}{\usepackage{xurl}}{} % add URL line breaks if available
\IfFileExists{bookmark.sty}{\usepackage{bookmark}}{\usepackage{hyperref}}
\hypersetup{
  pdftitle={Distribution and Validation of EPSPS- and PPO-Inhibitor Resistance in Amaranthus palmeri from Southwestern Nebraska.},
  hidelinks,
  pdfcreator={LaTeX via pandoc}}
\urlstyle{same} % disable monospaced font for URLs
\usepackage[margin=1in]{geometry}
\setlength{\emergencystretch}{3em} % prevent overfull lines
\providecommand{\tightlist}{%
  \setlength{\itemsep}{0pt}\setlength{\parskip}{0pt}}
\setcounter{secnumdepth}{5}
% Allowing for landscape pages
\usepackage{lscape}
\usepackage{booktabs}
\usepackage{multirow}
\usepackage{float}
\usepackage{placeins}
\newcommand{\blandscape}{\begin{landscape}}
\newcommand{\elandscape}{\end{landscape}}
\usepackage{lineno}
\usepackage{setspace}

% Left justification of the text: see https://www.sharelatex.com/learn/Text_alignment
% \usepackage[document]{ragged2e} % already in the latex template
\newcommand{\bleft}{\begin{flushleft}}
\newcommand{\eleft}{\end{flushleft}}

\title{Distribution and Validation of EPSPS- and PPO-Inhibitor Resistance in
\emph{Amaranthus palmeri} from Southwestern Nebraska.}
\author{true \and true \and true \and true \and true}
\date{}

\begin{document}
\maketitle

\vspace{2mm}

\doublespace

\large Abstract

\textbf{BACKGROUNG:} Failure to control \textit{Amaranthus palmeri} with
5-enolpyruvylshikimate-3-phosphate synthase (EPSPS)- and
protoporphyrinogen oxidase (PPO)-inhibitor herbicides was reported
across southwestern Nebraska in 2017. The objectives of this study were
to (1) confirm and validate EPSPS (glyphosate)- and PPO (fomesafen and
lactofen)-resistance in 51 \textit{A. palmeri} populations from
southwestern Nebraska using genotypic and whole-plant phenotypic assays,
and (2) determine which agronomic practices are influencing
EPSPS-inhibitor resistance in \textit{A. palmeri} populations in that
geography.\\
\textbf{RESULTS:} Based on genotypic assay, 88\% of 51 populations
contained at least one individual with \textit{EPSPS} gene amplification
(\textgreater{} 2 copies), which confers glyphosate resistance; or a
mutation in the \textit{PPX2} gene \(\triangle\)G210 or R128G/M,
endowing PPO-inhibitor resistance in \textit{A. palmeri}. High
correlation (0.83) between genotypic and phenotypic assays demonstrated
that \textit{EPSPS} gene amplification is the main glyphosate resistance
mechanism in \textit{A. palmeri} populations from southwesten Nebraska.
In contrast, there was poor association between genotypic and phenotypic
responses for PPO resistance, which we attribute to a combination of
other resistance mechanisms being present, segregation of the
populations, and using too high of herbicide doses. Thus, genotypic
assays could expedite the process for confirmation of EPSPS- but not
necessarily for PPO-inhibitor resistance in \textit{A. palmeri} from
southwestern Nebraska. Moreover, \textit{EPSPS} gene amplification,
county, current and previous crops are the main factors influencing
glyphosate resistance within that geography. Most of EPSPS-susceptible
\textit{A. palmeri} populations were found in a few counties in high
crop diversity areas.\\
\textbf{CONCLUSIONS:} Results presented here confirmed the spread of
EPSPS- and PPO-inhibitor resistance in \textit{A. palmeri} populations
from southwestern Nebraska and show that less diverse cropping systems
are an important driver for EPSPS-inhibitor resistance evolution in
\textit{A. palmeri}.

\vspace{2mm}

\noindent \textbf{Keywords}: Glyphosate; Fomesafen; Lactofen; Palmer
amaranth; Weed resistance.

\linenumbers

\hypertarget{introduction}{%
\section{Introduction}\label{introduction}}

\emph{Amaranthus palmeri} S. Watson (Palmer amaranth) is a plant species
indigenous to the southwestern United States (US) and northern
Mexico.\textsuperscript{1} Despite being an edible plant used to feed
humans and animals,\textsuperscript{2} \emph{A. palmeri} has long been
documented as a serious weed problem in US cropping
systems.\textsuperscript{3} \emph{A. palmeri} produces thousands of
seeds and grows up to 2 m tall with many lateral
branches,\textsuperscript{4} making it a very competitive species with
crops.\textsuperscript{5,6} In the 1970s, when cotton picking became
mechanized, machinery contributed to the spread of \emph{A. palmeri}
seeds across the southern United States.\textsuperscript{7} At that
time, \emph{A. palmeri} was considered the most successful weed of all
dioecious \emph{Amaranthus} species as it became widespread in cotton
fields.\textsuperscript{7} Currently, \emph{A. palmeri} is the most
economically damaging weed species infesting corn, cotton, and soybean
fields in the southern US.\textsuperscript{8,9}

obligate cross-pollination in A. palmeri is herbicide resistance
movement via gene flow within and among populations within the species

The economic importance of \emph{A. palmeri} is primarily related to its
ability to evolve resistance to herbicides. Factors relating to the
intrinsic biology of this species have contributed to its fast herbicide
resistance evolution.\textsuperscript{9} For example, \emph{A. palmeri}
reproduces via obligate cross pollination, increasing the chances of
herbicide resistance alleles transfer via gene flow within and among
populations.\textsuperscript{10,11} In addition, human-driven selection
has strongly contributed to the rise of \emph{A. palmeri} as a
problematic weed. In the US cropping systems, no-till is a widely used
practice amongst growers, and \emph{A. palmeri} thrives in no-till
fields due to its small seed size, contributing to the rapid increase of
\emph{A. palmeri} individuals in field crops. Also, herbicide resistance
in \emph{A. palmeri} drastically increased when weed management
strategies shifted from the use of multiple herbicide sites of action
(SOA) in a single season to reliance on a single SOA post-emergence
herbicide (e.g., glyphosate).\textsuperscript{12} Thus far, \emph{A.
palmeri} has evolved resistance to herbicide resistance to multiple
SOA,\textsuperscript{13} which is a major concern as weed management in
conventional US cropping systems is largely herbicide dependent.

The history of herbicide resistance evolution in \emph{A. palmeri} is a
result of intense selection pressure from herbicides. In the 1990s, the
first documented cases of herbicide resistance were against
microtubule-,\textsuperscript{14} ALS-,\textsuperscript{15} and
PSII-inhibitor herbicides.\textsuperscript{13} With the introduction of
glyphosate-resistant (GR) crops, post-emergence applications of
glyphosate became widely used for weed management in soybean, cotton,
and corn, resulting in evolution of GR \emph{A.
palmeri}.\textsuperscript{16} GR \emph{A. palmeri} spread across the
southern and midwestern US through both independent herbicide
selection\textsuperscript{17} and seed dispersal.\textsuperscript{18,19}
The spread of GR \emph{A. palmeri} led to a reevaluation of the use of
glyphosate as a sole means of weed control in this species, and a push
to diversify weed management strategies (e.g., application of other
herbicide SOA). The use of 4-hydroxyphenylpyruvate dioxygenase (HPPD)-,
protoporphyrinogen oxidase (PPO)- and long chain fatty acids (LCFA)-
inhibitor herbicides increased in an attempt to manage GR \emph{A.
palmeri}. However, \emph{A. palmeri} populations also evolved resistance
to HPPD-, PPO- and LCFA-inhibitor herbicides.\textsuperscript{13} New
technologies such as auxin-resistant crops may be jeopardized by the
newest reports of 2,4-D-resistant \emph{A. palmeri}\textsuperscript{13}
and the number of \emph{A. palmeri} populations with herbicide
resistance to multiple herbicide resistance is also on the
rise.\textsuperscript{20,21} Therefore, \emph{A. palmeri} herbicide
resistance evolution is shrinking the chemical control options for weed
management in corn, soybean, and cotton fields within US cropping
systems.

In the US Midwest, corn and soybean growers strongly rely on EPSPS-
(e.g., glyphosate) and PPO- (e.g., fomesafen) inhibitor herbicides for
weed management. The recent migration of \emph{A. palmeri} into the US
Midwest poses a serious threat to the sustainability of crop production
in that geography.\textsuperscript{{\textbf{???}},22} \emph{A. palmeri}
is now overlapping territory with another problematic dioecious
\emph{Amaranthus} species, waterhemp (\emph{Amaranthus tuberculatus}).
Thus, prevention and/or rapid diagnosis of herbicide resistance in
\emph{A. palmeri} has become a priority for agricultural stakeholders.
The advances in high-throughput genome sequencing methods are expediting
the elucidation and detection of herbicide resistance mechanisms in
\emph{A. palmeri} and other weed species. In the case of glyphosate, the
most common resistance mechanism in \emph{A. palmeri} is \emph{EPSPS}
gene amplification,\textsuperscript{23,24} while in PPO-inhibitor
resistance, the major resistance mechanism is a \emph{PPO2} glycine 210
deletion (\(\triangle\)G210).\textsuperscript{25,26} Nonetheless, novel
herbicide resistance mechanisms in \emph{A. palmeri} are still being
uncovered,\textsuperscript{23} as evidenced by the recent documentation
of two mutations in the \emph{PPO2} enzyme in the R128 site of \emph{A.
palmeri},\textsuperscript{27} and G399A, an amino acid substitution of
glycine to alanine in the catalytic domain of \emph{PPO2} at position
399.\textsuperscript{28} So far, a few reports of resistance mechanisms
have been confirmed for \emph{A. palmeri} populations with
non-target-site resistance to
glyphosate\textsuperscript{{\textbf{???}},29} and one against
PPO-inhibitor herbicides.\textsuperscript{30} Therefore, using genotypic
assays might provide faster detection of known herbicide weed resistance
mechanisms in \emph{A. palmeri}, but it fails to address herbicide
resistance resulting from novel mechanisms, including polygenic
resistance mechanisms (e.g.~metabolic resistance involving cytochrome
P450 genes) as well as previously unknown target site mutations.

In the fall of 2017, growers in southwestern Nebraska reported failure
to control \emph{A. palmeri} with EPSPS- and PPO-inhibitor herbicides
(Werle R, personal communication). The objectives of this study were to
(1) confirm EPSPS- and PPO-inhibitor resistance in 51 \emph{A. palmeri}
populations from southwestern Nebraska via genotypic resistance assays
and compare these results via whole-plant phenotypic assay of their
progenies, and (2) evaluate agronomic practices that may contribute to
EPSPS- and PPO-inhibitor resistance in \emph{A. palmeri} populations.

\hypertarget{material-and-methods}{%
\section{Material and Methods}\label{material-and-methods}}

\hypertarget{plant-material-and-growing-conditions}{%
\subsection{Plant Material and Growing
Conditions}\label{plant-material-and-growing-conditions}}

The study was performed with 51 arbitrarily selected \emph{A. palmeri}
populations infesting cropping systems across Southwestern Nebraska.
Each population was collected from a single field. Location, agronomic
practices, \emph{A. palmeri} distribution and density of each population
were collected (Table 1). In August 2017, green leaf tissues were
harvested from five actively growing plants from each of the 51 \emph{A.
palmeri} populations, then labeled and stored at -80 C to be used in
genotypic assays. Within the 51 \emph{A. palmeri} populations, a second
sampling of 19 arbitrarily selected populations was obtained by
collecting seeds (progenies) from 30 plants each in September 2017, then
cleaned, bulked for each population, and stored at 5 C until the onset
of the greenhouse experiments. Seeds were planted in 900
cm\textsuperscript{-3} plastic trays containing potting-mix
(Pro-Mix\textsuperscript \textregistered, Quakertown, PA, USA). Emerged
seedlings (1 cm) were transplanted into 164 cm\textsuperscript{-3}
cone-tainers. \emph{A. palmeri} plants were supplied with adequate water
and kept under greenhouse conditions at 28/20 C day/night temperature
with 80\% relative humidity. Artificial lighting was provided using
metal halide lamps (600 \(\mu\)mol m\textsuperscript{-2}
s\textsuperscript{-1}) to ensure 15 h photoperiod.

\hypertarget{genotypic-herbicide-resistance-mechanism-assays}{%
\subsection{Genotypic Herbicide Resistance Mechanism
Assays}\label{genotypic-herbicide-resistance-mechanism-assays}}

As a standard methodology from the University of Illinois Plant Clinic
Laboratory (Urbana,85IL 61801, United States), three to five leaf tissue
samples were collected from 51 \emph{A. palmeri} populations from
Southwestern Nebraska. The EPSPS copy number was estimated for each
plant based on DNA extracted from tissue from a single leaf of \emph{A.
palmeiri}. Genomic DNA extraction from leaf tissue87samples were
performed using a modified CTAB method.\textsuperscript{31} DNA quality
and quantity were checked on a Nanodrop 1000 (Thermo Fisher Scientific,
Inc., Waltham, MA, USA) and any samples with low DNA yields or high
protein:DNA ratios were discarded and re-extracted. TaqMan qPCR assays
were used to check for the presence of known PPO-inhibitor resistance
mutations on the \emph{PPO2} enzyme, including the glycine 210
deletion\textsuperscript{32} and the R128G/M
mutations.\textsuperscript{33} Samples were also tested for glyphosate
resistance via increased numbers of \emph{EPSPS} genomic copies using a
SYBR qPCR approach\textsuperscript{34} in which \emph{EPSPS} copy
numbers were estimated based on comparison with a single-copy reference
gene (\emph{CPS}, carbamoyl phosphate synthetase).

Herein, for the objective of this study, individuals possessing
\emph{EPSPS} copy numbers \textgreater{} 2 are considered
EPSPS-resistant and individuals with presence of \(\triangle\)G210 or
R128 het mutations are considered PPO-resistant. Other target-site and
non-target site resistance mechanisms were not tested.

\hypertarget{whole-plant-phenotypic-assay}{%
\subsection{Whole-plant Phenotypic
Assay}\label{whole-plant-phenotypic-assay}}

The research was conducted under greenhouse conditions in 2018 and 2019
at the University of Wisconsin-Madison to evaluate the sensitivity of 19
\emph{A. palmeri} populations from Southwestern Nebraska to EPSPS- and
PPO-inhibitor herbicides.

The experiments were conducted in a complete randomized design and the
experimental unit was a cone-tainer with a single \emph{A. palmeri}
seedling. The study was arranged in a factorial design with \emph{A.
palmeri} progenies from 19 populations and three herbicides with 20
replications and conducted twice. Altogether, 2280 \emph{A. palmeri}
seedlings were screened in the phenotypic assay. The arbitrarily
selected 19 \emph{A. palmeri} progenies were from Cha 3, Dun 3, Dun 4,
Dun 5, Hay 1, Hay 3, Hay 4, Kei 2, Kei 3, Kei 5, Kei 6, Log 1, Log 2,
Log 4, Per 2, Per 4, Red 2, Red 4, and Red 5 populations (Table 1). The
selected herbicides were glyphosate (Roundup
PowerMAX\textsuperscript \textregistered, Bayer Crop Science, Saint
Louis, MO, US) applied at 870 g ae ha\textsuperscript{-1} plus 2040 g
ha\textsuperscript{-1} ammonium sulfate (DSM Chemicals North America
Inc., Augusta, GA); fomesafen (Flexstar\textsuperscript \textregistered,
Syngenta Crop Protection, Greensboro, NC, USA) applied at 206 g ai
ha\textsuperscript{-1} plus 0.5 L ha\textsuperscript{-1} of non-ionic
surfactant (Induce\textsuperscript \textregistered, Helena
Agri-Enterprises, Collierville, TN, USA); and lactofen
(Cobra\textsuperscript \textregistered, Valent USA LLC Agricultural
Products, Walnut Creek, CA, USA) applied at 280 g ai
ha\textsuperscript{-1} plus 0.5 L ha\textsuperscript{-1} of non-ionic
surfactant.

Herbicide treatments were applied to 8-10 cm tall \emph{A. palmeri}
plants with a single-tip chamber sprayer (DeVries Manufacturing Corp.,
Hollandale, MN, USA). The sprayer had an 8001 E nozzle (Spraying Systems
Co., North Avenue, Wheaton, IL, US) calibrated to deliver 140 L
ha\textsuperscript{-1} spray volume at 135 kPa at a speed of 2.3 km
h\textsuperscript{-1}. \emph{A. palmeri} populations were visually
assessed 21 days after treatment (DAT) as dead or alive. Plants were
considered alive when prominent green tissue was observed in growing
plants, while dead plants were completely necrotic.

\hypertarget{statistical-analysis}{%
\subsection{Statistical Analysis}\label{statistical-analysis}}

\hypertarget{genotypic-and-phenotypic-validation-of-epsps--and-ppo-inhibitor-resistance-in-a.-palmeri}{%
\subsubsection{\texorpdfstring{Genotypic and Phenotypic validation of
EPSPS- and PPO-inhibitor resistance in \emph{A.
palmeri}}{Genotypic and Phenotypic validation of EPSPS- and PPO-inhibitor resistance in A. palmeri}}\label{genotypic-and-phenotypic-validation-of-epsps--and-ppo-inhibitor-resistance-in-a.-palmeri}}

The number of EPSPS- or PPO-inhibitor-resistant \emph{A. palmeri}
individuals in the genotypic assays was converted to a percentage scale:

\emph{Equation 1}: \[G=\frac{S}{T} * 100 \] \noindent where \emph{G} is
the \% EPSPS- or PPO-inhibitor-resistant \emph{A. palmeri} individuals,
\emph{S} is the total number of \emph{A. palmeri} individuals positive
for herbicide resistance, and \emph{T} is the total number of \emph{A.
palmeri} individuals (n=3 to 5) screened for herbicide resistance in
genotypic assays. Fomesafen and lactofen are PPO-inhibitor herbicides;
thus, \emph{G} is same for both.

The number of surviving \emph{A. palmeri} individuals in the phenotypic
assay were converted into a percentage scale:

\emph{Equation 2}: \[P=\frac{X}{T} * 100 \]

Where \emph{P} is the \% alive \emph{A. palmeri} individuals after
herbicide treatment in phenotypic assay (glyphosate, fomesafen, or
lactofen), \emph{X} is the total number of alive \emph{A. palmeri}
individuals 21 DAT and \emph{T} is the total number of \emph{A. palmeri}
individuals (n=40) treated with each herbicide. The P (\%) can be
determined only for 19 populations, as only 19 populations were screened
to herbicides. Data of two runs are combined.

The correlation between \emph{G} and \emph{P} for each herbicide
(glyphosate, fomesafen, and lactofen) and between the two PPO-inhibitor
herbicides, fomesafen and lactofen, were performed with Pearson's
analysis using the \emph{cor.test} function of R statistical software
version 3.6.1.\textsuperscript{35} The correlation value varies from -1
and 1, where 1 is the total positive correlation, -1 the total negative
correlation, and 0 indicate no linear correlation. Pearson's analysis
tests the hypothesis that correlation between two variables is equal to
0 (null hypothesis). If \emph{P}-value \textgreater{} 0.05, the
probability is \textgreater{} 5\% that a correlation of some magnitude
between two variables could occur by chance alone assuming null
hypothesis is true; thus, no correaltion between variables. The \emph{G}
and \emph{P} correlation was performed with the 19 \emph{A. palmeri}
populations treated with herbicide in the phenotypic assay as well as
their respective genotypic assays results.

The individuals screened in genotypic and phenotypic assays are parent
and progenies, respectively. This methodology was chosen to simulate a
real farm scenario whereas growers could collect suspected herbicide
resistant leaf samples and mail them to a laboratory for faster
genotypic herbicide resistance confirmation.

\hypertarget{random-forest}{%
\subsubsection{Random Forest}\label{random-forest}}

Random forest is a powerful ensemble machine learning algorithm which
generates and combines multiple decision trees to reach a consensus. The
random forest procedure is described in detail by
Breiman\textsuperscript{36} and Biau.\textsuperscript{37} In short, the
random forest analysis is largely based on two parameters: \emph{ntree},
which is the number of decision trees, and \emph{mtry}, the number of
different predictors tested in each tree. For each decision tree, a
subsample of observations from the data are selected with replacement to
train the trees (bootstrap aggregating). These ``in-bag'' samples
include approximately 66\% of the total data and some observations may
be repeated in each new training data set since this sampling is done
with replacement. The remaining 33\% of the data are designated
``Out-of-Bag'' or OOB samples and are used in an internal
cross-validation technique to estimate the model performance error. To
evaluate the importance of an explanatory variable (or predictor), the
random forest measures both the decrease in model performance accuracy
as calculated by the OOB error and the decrease in the Gini Index value.
The Gini Index value (mean decrease in accuracy) is the mean of a total
variable decrease of a node impurity, weighted by the samples proportion
reaching that node in each individual decision tree. Therefore,
variables with a large Gini Index value indicates higher variable
importance, and are more important for data classification. Random
Forest has been used to described the incidence of crop
disease\textsuperscript{38} and glyphosate resistance in
\emph{Amaranthus} spp..\textsuperscript{39}

The random forest analysis was performed with the \emph{randomForest}
package of R statistical software to describe the influence of
\emph{EPSPS} gene amplification (genotypic), PPO-inhibitor resistance
(genotypic), location (county), agronomic practices (e.g., tillage,
irrigation, current and previous cropping-system), and weed demographics
(e.g., density and distribution) on EPSPS-inhibitor resistance in
\emph{A. palmeri} in southwestern Nebraska (Table 1). \emph{EPSPS} gene
copy number (genotypic) was included as an explanatory variable to test
the robustness of random forest as it is highly correlated to glyphosate
resistance in \emph{A. palmeri}.\textsuperscript{23} The \emph{A.
palmeri} populations with at least one individual with \textgreater{} 2
\emph{EPSPS} copy number was considered resistant (Table 2); therefore,
EPSPS-inhibitor resistance was classified as Yes (\textgreater{} 2
\emph{EPSPS} copy number) or No.~For this analysis, the \emph{ntree}
parameter was set to 500, whereas \emph{mtry} was set to 2. Parameters
numbers selected are standard for the sample size of the present study.

\hypertarget{results}{%
\section{Results}\label{results}}

\hypertarget{genotypic-and-phenotypic-validation-of-epsps--and-ppo-inhibitor-resistance-in-a.-palmeri-1}{%
\subsection{\texorpdfstring{Genotypic and phenotypic validation of
EPSPS- and PPO-inhibitor resistance in \emph{A.
palmeri}}{Genotypic and phenotypic validation of EPSPS- and PPO-inhibitor resistance in A. palmeri}}\label{genotypic-and-phenotypic-validation-of-epsps--and-ppo-inhibitor-resistance-in-a.-palmeri-1}}

The \emph{EPSPS} amplification was found in 63\% of the 51 \emph{A.
palmeri} populations analysed (Table 2). Phenotypic analysis of a subset
of these populations confirmed the genotypic analysis data, in that a
positive correlation (0.83; \emph{P}-value=0.0000) was observed between
\emph{G} and \emph{P} assays (Figure 1 and Table 3). Seven \emph{A.
palmeri} populations tested negative (\emph{G}=0) for glyphosate
resistance in the genotypic assay but three of these populations showed
low (\emph{P}=18\%, Hay 1), moderate (\emph{P}=35\%, Hay 4) and high
(\emph{P}=75\%, Red 5) survival after glyphosate treatment in the
phenotypic assays (Figure 1). The other four populations (Dun 3, Hay 3,
Log 2 and Kei 3) that tested negative (\emph{G}=0\%) for EPSPS-inhibitor
resistance in the genotypic assay showed less than 15\% glyphosate
survival (\emph{P}).

Fomesafen and lactofen resulted in less than 40\% survival of \emph{A.
palmeri} populations in the phenotypic assays (Figure 2). The
correlation between \emph{G} and \emph{P} for PPO-inhibitor resistance
in \emph{A. palmeri} populations was inconsistent (Table 3). While a
high \emph{G} and \emph{P} correlation (0.52; \emph{P}-value=0.0217) was
observed for fomesafen (Figure 2A), no \emph{G} and \emph{P} correlation
(-0.05; \emph{P}-value=0.84) was found for lactofen (Figure 2B).
\emph{A. palmeri} populations Dun 5, Kei 2, Kei 5, and Log 4 were
\emph{segregating for PPO resistance (Supp Table)} in the genotyping
assay but individuals in these populations were sensitive to lactofen
treatment (\emph{P}=0\%, Figure 2B). However, these populations showed
higher frequency of resistant individuals. For example, nearly 30\% of
individuals from the Log 4 population survived fomesafen treatment
(Figure 2A). In contrast, \emph{A. palmeri} populations Cha 3, Kei 6,
Per 2, and Red 5 tested negative for \(\triangle\)G210 or R128 het
mutations (\emph{G}=0\%) but over 15\% of populations survived both
fomesafen and lactofen treatment. Also, \emph{A. palmeri} populations
Kei 3, Per 4, and Dun 4 showed 38, 25, and 18\% survival after fomesafen
treatment but less than 15\% for lactofen. There was no correlation
(0.23; \emph{P}-value=0.34) between fomesafen and lactofen in the
phenotypic assay (Table 3).

According to the genotypic assay, nearly 70\% of the 51 \emph{A.
palmeri} populations from Southwestern Nebraska were confirmed resistant
to PPO-inhibitor herbicides. In the 34 PPO-inhibitor-resistant \emph{A.
palmeri} populations, 32 contain the \(\triangle\)G210 deletion in the
\emph{PPX2} gene, while the R128M/G endowed PPO-inhibitor resistance in
two \emph{A. palmeri} populations. Nearly 14\% of \emph{A. palmeri}
populations had all individuals resistant, 53\% were segregating for
resistance, and 33\% had no mutation detected to PPO-inhibitor
herbicides. In addition, based on \emph{EPSPS} gene amplification, our
study showed that 10\% of \emph{A. palmeri} populations had all
individuals resistant to glyphosate, 53\% were segregating for
resistance, and 37\% were susceptible to glyphosate (Table 2). Multiple
resistance (EPSPS- and PPO-inhibitor herbicides) was present in 41\% of
\emph{A. palmeri} populations of southwestern Nebraska (Table 2), while
6, 11, and 13 \emph{A. palmeri} populations were susceptible to both
herbicides, resistant to only glyphosate-, and resistant to only
PPO-inhibitor herbicides, respectively (Figure 3).

\hypertarget{random-forest-1}{%
\subsection{Random Forest}\label{random-forest-1}}

The final OOB error rate of the random forest analysis was 13.33\%,
meaning that \textgreater{} 86\% of OOB samples were adequately
classified by the model. The random forest analysis ranked (high to low)
\emph{EPSPS} gene amplification \textgreater{} county \textgreater{}
current crop \textgreater{} previous crop \textgreater{} \emph{A.
palmeri} density \textgreater{} tillage \textgreater{} irrigation
\textgreater{} \emph{A. palmeri} distribution \textgreater{}
PPO-inhibitor resistance as the factors influencing the presence of
EPSPS-inhibitor resistance in \emph{A. palmeri} of southwestern Nebraska
(Figure 4). The selection of \emph{EPSPS} gene amplification as the top
predictor shows the robustness of this model, since \emph{EPSPS} gene
amplification is known to confer resistance to glyphosate. County was
the second most important factor for the presence of EPSPS-inhibitor
resistant \emph{A. palmeri}. The presence of EPSPS-inhibitor resistance
in \emph{A. palmeri} was found in at least one population of all
surveyed counties from southwestern Nebraska (Table 2). The lowest
number of EPSPS-inhibitor-resistant populations was found at Hayes (Hay)
and Perkins (Per) County with 1 (out 5) and 2 (out 6), respectively.
Also, current and previous crops strongly influenced the presence of
EPSPS-inhibitor resistance in \emph{A. palmeri} populations. Five
\emph{A. palmeri} populations (Dun 4, Dun 5, Fro 1, Kei 2, and Lin 2)
had 100\% resistant individuals and these populations were all found in
current corn or soybean crops preceded by a corn or sorghum crops
(Figure 5A). In contrast, \emph{EPSPS} gene amplification did not occur
in 19 \emph{A. palmeri} populations, from which only two populations
were found in corn and soybean rotations (Fro 5 and Hit 5; Table 2). The
majority of EPSPS-inhibitor-susceptible \emph{A. palmeri} populations
were found in rotations of corn, soybean, sorghum, wheat, fallow, and
other crops (e.g., alfalfa, dry beans and field peas; Figure 5A).

Nearly 70\% of EPSPS-inhibitor-susceptible \emph{A. palmeri} populations
were resistant to PPO-inhibitor herbicides (Table 2). A similar trend of
crop diversity was observed for PPO-inhibitor resistance in \emph{A.
palmeri}, except a crop rotation of corn-sorghum-soybean was found in
locations with only PPO-inhibitor-resistant individuals (Figure 5B). The
random forest analysis was performed on EPSPS-inhibitor resistance only
due to the robustness of \emph{EPSPS} gene amplification as a positive
control to detect resistance in \emph{A. palmeri}, which is different
from our results with PPO-inhibitor resistance.

\hypertarget{discussion}{%
\section{Discussion}\label{discussion}}

The high correlation between \emph{G} and \emph{P} for EPSPS-inhibitor
resistance demonstrates that most GR \emph{A. palmeri} populations from
southwestern Nebraska are resistant due to \emph{EPSPS} gene
amplification. However, other resistance mechanisms are likely present
since a few \emph{A. palmeri} populations showed no \emph{EPSPS} gene
amplification despite a majority of the individuals survived glyphosate
(870 g ae ha\textsuperscript{-1}) application in the phenotypic assay.
\emph{A. palmeri} was the first identified weed to evolve glyphosate
resistance via \emph{EPSPS} gene amplification,\textsuperscript{40}
followed by \emph{Kochia scoparia}, \emph{Amaranthus tuberculatus},
\emph{Lolium perenne ssp. multiflorum}, \emph{Bromus diandrus},
\emph{Eleusine indica}, \emph{Chloris truncata}, and \emph{Amaranthus
hybridus}.\textsuperscript{41,42} The \emph{EPSPS} gene amplification
mechanism is widespread in \emph{A. palmeri},\textsuperscript{23,41}
albeit other EPSPS-inhibitor resistance mechanisms have arisen,
including \emph{Pro106} mutation in the \emph{EPSPS} gene and reduced
glyphosate
absorption/translocation.\textsuperscript{{\textbf{???}},29,41} It
remains unknown whether the \emph{A. palmeri} populations (e.g., Red 5)
which have low or no \emph{EPSPS} gene amplification harbor additional
resistance mechanisms, warranting further investigations.

Gene amplification is an important evolutionary mechanism enabling weeds
and other pests\textsuperscript{43,44} to evolve resistance to
pesticides. Research on \emph{EPSPS} gene amplification molecular basis
in weed species is underway but more work is still needed to unveil this
complex adaptative trait. The genetics of \emph{EPSPS} gene
amplification in weed species follows Mendelian (\emph{K.
scoparia})\textsuperscript{45} and non-Mendelian (\emph{A.
palmeri}\textsuperscript{46} and \emph{B. diandrus}\textsuperscript{47})
inheritance patterns. Over 100 \emph{EPSPS} gene copies have been
documented in \emph{A. palmeri} but maximum of only 10 have been
observed in \emph{K. scoparia}.\textsuperscript{48,49} The \emph{EPSPS}
gene copy variation in \emph{A. palmeri} is a result of the
extrachromosomal circular DNA transmitted to the next generation by
tethering to mitotic and meiotic chromosomes
(eccDNA),\textsuperscript{50} while in \emph{K. scoparia}, \emph{EPSPS}
copies are arranged in tandem repeats at a single
locus.\textsuperscript{42} Segregation for \emph{EPSPS} copy number in
\emph{A. palmeri} families (F\scriptsize 1 \normalsize and
F\scriptsize 2 \normalsize) is transgressive, with individuals varying
in \emph{EPSPS} gene amplification levels even among clonal
plants.\textsuperscript{46} Transgressive segregation for \emph{EPSPS}
in \emph{A. palmeri} might explain the variable \emph{EPSPS} copy
numbers between individuals within populations screened from
southwestern Nebraska (Table 2). Gene amplification coupled with the
dioecious nature of \emph{A. palmeri} are valuable traits for this weed
that help increase its genetic complexity and allow it to adapt to
current US cropping systems.

According to the genotypic assay, most of \emph{A. palmeri} populations
from southwestern Nebraska also showed resistance to PPO-inhibitor
herbicides. The mechanism of resistance is largely due to the
\(\triangle\)G210 deletion but also due to R128M/G mutations in the
\emph{PPX2} gene (Table 2). It has been demonstrated that a mutated PPO
enzyme has reduced affinity for several PPO-inhibitor herbicides in
\emph{A. palmeri}.\textsuperscript{20} Application of fomesafen (206 g
ai ha\textsuperscript{-1}) and lactofen (280 g ai
ha\textsuperscript{-1}) provided high mortality in \emph{A. palmeri}
populations from southwestern Nebraska, including \emph{A. palmeri}
populations with 100\% individuals with \(\triangle\)G210 deletion.
Therefore, it is possible that most of sprayed \emph{A. palmeri}
individuals in the phenotypic assay had no mutation for PPO-inhibitor
resistance. High \emph{A. palmeri} mortality suggests that populations
are segregating for PPO-inhibitor resistance at a low level. Still,
there is correlation between \emph{G} and \emph{P} for fomesafen but not
lactofen, which was sprayed slightly above the labled rate of 218 g ai
ha\textsuperscript{-1}. The validation for PPO-inhibitor resistance
presented here is limited by the segregating nature of \emph{A. palmeri}
collected from field populations, since individuals in the phenotypic
assays were not the same as those screened in the genotypic assays. This
could be also explained in several ways, 1) the number of individuals
sampled for the \emph{G} assay study may have been too low for the
objective of validation; 2) high herbicide rate resulting in high
individual mortality; 3) greenhouse conditions were ideal and plants had
no stress during application of PPO-inhibitor herbicides, which might be
different from plants under field conditions of southwestern
Nebraska;\textsuperscript{51} and 4) it is likely that plants were
smaller than usual because of small volume of cone-tainers, the size of
a plant strongly impacts the level of resistance, with smaller plants
showing less resistance than larger plants.\textsuperscript{52}

Populations of \emph{A. palmeri} have been reported to be resistant to
HPPD-\textsuperscript{53}, PSII-\textsuperscript{53} and EPSPS-inhibitor
herbicides\textsuperscript{22} in Nebraska. In 2014, a survey with
\emph{Amaranthus} spp. in southwestern Nebraska showed widespread
EPSPS-inhibitor resistance for \emph{A. tuberculatus} (81\%) but not for
\emph{A. palmeri} (6\%).\textsuperscript{39} Our survey showed a high
number of EPSPS-inhibitor-resistant/segregating \emph{A. palmeri}
populations in southwestern Nebraska (\textgreater60\%). The rapid
evolution of EPSPS-inhibitor-resistant \emph{A. palmeri} in southwestern
Nebraska raised questions about whether populations were introduced via
seed/gene flow or arose independently. Although we did not test the
hypothesis, the random forest analysis did shed some light on
EPSPS-inhibitor resistance evolution in \emph{A. palmeri}. The random
forest analysis results suggested that \emph{EPSPS} gene amplification,
county, and current and previous crops were the top factors influencing
GR \emph{A. palmeri} in southwestern Nebraska. The \emph{EPSPS} gene
amplification as the primary factor for glyphosate resistance in
\emph{A. palmeri} showed the robustness of random forest analysis, while
county influence on \emph{EPSPS}-inhibitor resistance in \emph{A.
palmeri} highlighted how practices for delaying EPSPS-inhibitor
resistance evolution should start at the landscape level. It is likely
that growers with common goals may implement best weed management
tactics to delay EPSPS-inhibitor resistance evolution and prevent
seed/gene flow of herbicide resistant weeds. Nonetheless, it is striking
that PPO-inhibitor resistance was common in counties with low/no
EPSPS-inhibitor resistance. For instance, most
EPSPS-inhibitor-susceptible \emph{A. palmeri} were found in Hayes County
in areas of sorghum or corn in rotation with wheat (Table 1), which are
less glyphosate-dependent crops. Although best weed management practices
at the landscape level are encouraged, our results suggest that growers
were not necessarily implementing good management practices but rather,
using less EPSPS-inhibitor herbicide.

The high presence of \emph{A. palmeri} populations with 100\% resistance
to EPSPS-inhibitors in less diverse cropping systems (e.g., corn-soybean
rotation) suggests the influence of repeated glyphosate application on
\emph{A. palmeri} evolution to EPSPS-inhibitor resistance. The
occurrence of EPSPS-inhibitor-resistant \emph{A. palmeri} populations is
reduced in rotations with more diversified crops (Figure 5A), most
likely due to rotations with glyphosate sensitive crops (no/less
glyphosate use). Crop diversity exerts a different selection pressure on
weed communities, including canopy closure timing, seeding, and harvest
date, which aid on reducing the dominance of single weed
species.\textsuperscript{54} Nonetheless, it is likely that Nebraska
growers still rely on other herbicide sites of action for weed
management, as most EPSPS-inhibitor-susceptible \emph{A. palmeri}
populations were resistant to PPO-inhibitor herbicides (e.g.,
fomesafen). Similar to EPSPS-inhibitor resistance, the
PPO-inhibitor-resistant \emph{A. palmeri} was also found more often in
less diverse crop rotations (Figure 5B). Despite not having long-term
herbicide application records for the surveyed area, it is possible that
overreliance on a single or few herbicide SOA in areas with low crop
diversity contributed to resistance. In addition, it has been shown that
herbicide mixture (multiple SOA in one application) is more effective
for delaying herbicide weed resistance than herbicide rotation (multiple
applications, each with a single SOA).\textsuperscript{55,56} Thus, it
is likely that without herbicide mixture, increased crop diversity is
not enough to reduce the risk of herbicide resistance evolution in weed
species.

An emerging concern in weed science is the ability of some species to
stack genes for multiple herbicide resistance in a single population.
Five- and six-way herbicide resistance has been reported in an \emph{A.
tuberculatus} population from Illinois\textsuperscript{57} and
Missouri,\textsuperscript{58} respectively. The 19 \emph{A. palmeri}
populations used in the whole-plant phenotypic assay also tested
positive (\textgreater80\% of individuals) to an ALS-inhibitor herbicide
(imazethapyr {[}70 g ai ha\textsuperscript{-1}{]}; data not shown).
Moreover, nearly half of the \emph{A. palmeri} populations in this study
possessed multiple herbicide resistance (EPSPS and PPO; Figure 3). Thus,
it is likely that two- and three-way resistance exists in most \emph{A.
palmeri} populations from southwestern Nebraska. The random forest
analysis suggested no/low influence of PPO-inhibitor resistance on
EPSPS-inhibitor resistance, indicating \emph{EPSPS} gene amplification
and PPO-inhibitor resistance mutations (\(\triangle\)G210 or R128M/G)
are not genetically linked. Cosegregation of distinct SOA resistance
mechanisms is unlikely but not impossible as it has been shown that ALS-
and PPO-inhibitor resistance are genetically linked in a \emph{A.
tuberculatus} population.\textsuperscript{59}

\hypertarget{conclusion}{%
\section{Conclusion}\label{conclusion}}

Herein we demonstrate the widespread occurrence of EPSPS- and
PPO-inhibitor-resistant \emph{A. palmeri} in southwestern Nebraska.
Glyphosate resistance via \emph{EPSPS} gene amplification was present in
the majority of \emph{A. palmeri} populations from Nebraska, supporting
the use of genotypic assays for faster detection of
\emph{EPSPS}-inhibitor resistance in this region. Novel or non-target
glyphosate resistance mechanisms may also be present in few \emph{A.
palmeri} populations, but at a relatively low frequency. PPO-inhibitor
resistance was also present in these populations, but was less
correlated with known PPO-inhibitor resistance mutations, warranting the
use of both genotypic and whole-plant bioassays for testing
PPO-inhibitor resistance. The EPSPS- or PPO-inhibitor-resistant \emph{A.
palmeri} populations were found in areas of low crop diversity,
suggesting resistance evolution was mostly due to high selection
pressures of EPSPS- and PPO-inhibitor herbicides. Great progress has
been made towards understanding the \emph{A. palmeri} molecular basis of
resistance, but the continuous spread of herbicide resistance to new
geographies is evident. Thus, there is a need to change the current
methods for crop and weed management, including dependence on
corn/soybean rotation, use herbicide mixtures with different SOA rather
than sequential applications, and bring non-herbicide weed management
innovations to US cropping systems.

\hypertarget{acknowledgments}{%
\section{Acknowledgments}\label{acknowledgments}}

This research received no specific grant from any funding agency. We
also appreciate the help of Victor Ribeiro's assistance with the
greenhouse projects.

\hypertarget{conflicts-of-interest}{%
\section{Conflicts of Interest}\label{conflicts-of-interest}}

The authors declare no conflict of interest.

\hypertarget{references}{%
\section*{References}\label{references}}
\addcontentsline{toc}{section}{References}

\hypertarget{refs}{}
\leavevmode\hypertarget{ref-sauer_recent_1957}{}%
1 Sauer J, Recent migration and evolution of the dioecious Amaranths,
\emph{Evolution} \textbf{11}:11--31 (1957).

\leavevmode\hypertarget{ref-smith_fodder_1900}{}%
2 Smith J, Fodder and forage plants. Exclusive of the grasses,
Washington (1900).

\leavevmode\hypertarget{ref-hamilton_weeds_1958}{}%
3 Hamilton KC and Arle HF, Weeds od crops in Southern Arizona, College
of Agriculture, University of Arizona, Tucson, AZ (1958).

\leavevmode\hypertarget{ref-sauer_revision_1955}{}%
4 Sauer J, Revision of the dioecious Amaranths, \emph{Madroño}
\textbf{13}:5--46 (1955).

\leavevmode\hypertarget{ref-morgan_competitive_2001}{}%
5 Morgan GD, Baumann PA, and Chandler JM, Competitive impact of Palmer
amaranth (\emph{Amaranthus} \emph{Palmeri}) on cotton (\emph{Gossypium}
\emph{Hirsutum}) development and yield, \emph{Weed Technology}
\textbf{15}:408--412 (2001).

\leavevmode\hypertarget{ref-massinga_water_2003}{}%
6 Massinga RA, Currie RS, and Trooien TP, Water use and light
interception under Palmer amaranth (\emph{Amaranthus} \emph{Palmeri})
and corn competition, \emph{Weed Science} \textbf{51}:523--531 (2003).

\leavevmode\hypertarget{ref-sauer_dioecious_1972}{}%
7 Sauer JD, The dioecious Amaranths: A new species name and major range
extensions, \emph{Madroño} \textbf{21}:426--434 (1972).

\leavevmode\hypertarget{ref-price_glyphosate-resistant_2011}{}%
8 Price AJ, Balkcom KS, Culpepper SA, Kelton JA, Nichols RL, and
Schomberg H, Glyphosate-resistant Palmer amaranth: A threat to
conservation tillage, \emph{Journal of Soil and Water Conservation}
\textbf{66}:265--275 (2011).

\leavevmode\hypertarget{ref-ward_palmer_2013}{}%
9 Ward SM, Webster TM, and Steckel LE, Palmer Amaranth
(\emph{Amaranthus} \emph{Palmeri}): A Review, \emph{Weed Technology}
\textbf{27}:12--27 (2013).

\leavevmode\hypertarget{ref-oliveira_interspecific_2018}{}%
10 Oliveira MC, Gaines TA, Patterson EL, Jhala AJ, Irmak S, and Amundsen
K \emph{et al.}, Interspecific and intraspecific transference of
metabolism-based mesotrione resistance in dioecious weedy
\emph{Amaranthus}, \emph{The Plant Journal} \textbf{96}:1051--1063
(2018).

\leavevmode\hypertarget{ref-gaines_interspecific_2012}{}%
11 Gaines TA, Ward SM, Bukun B, Preston C, Leach JE, and Westra P,
Interspecific hybridization transfers a previously unknown glyphosate
resistance mechanism in \emph{Amaranthus} species, \emph{Evolutionary
Applications} \textbf{5}:29--38 (2012).

\leavevmode\hypertarget{ref-powles_evolved_2008}{}%
12 Powles SB, Evolved glyphosate-resistant weeds around the world:
Lessons to be learnt, \emph{Pest Management Science}
\textbf{64}:360--365 (2008).

\leavevmode\hypertarget{ref-heap_list_2019-1}{}%
13 Heap I, List of Herbicide Resistant Weeds by Weed Species, 2019.
\url{http://www.weedscience.org/Summary/Species.aspx} {[}accessed 12
June 2019{]}.

\leavevmode\hypertarget{ref-gossett_resistance_1992}{}%
14 Gossett BJ, Murdock EC, and Toler JE, Resistance of Palmer amaranth
(\emph{Amaranthus} \emph{Palmeri}) to the dinitroaniline herbicides,
\emph{Weed Technology} \textbf{6}:587--591 (1992).

\leavevmode\hypertarget{ref-horak_biotypes_1995}{}%
15 Horak MJ and Peterson DE, Biotypes of Palmer amaranth
(\emph{Amaranthus} \emph{Palmeri}) and common waterhemp
(\emph{Amaranthus} \emph{Rudis}) are resistant to imazethapyr and
thifensulfuron, \emph{Weed Technology} \textbf{9}:192--195 (1995).

\leavevmode\hypertarget{ref-culpepper_glyphosate-resistant_2006}{}%
16 Culpepper AS, Grey TL, Vencill WK, Kichler JM, Webster TM, and Brown
SM \emph{et al.}, Glyphosate-resistant Palmer amaranth
(\emph{Amaranthus} \emph{Palmeri}) confirmed in Georgia, \emph{Weed
Science} \textbf{54}:620--626 (2006).

\leavevmode\hypertarget{ref-kupper_population_2018}{}%
17 Küpper A, Manmathan HK, Giacomini D, Patterson EL, McCloskey WB, and
Gaines TA, Population genetic structure in glyphosate-resistant and
-susceptible Palmer amaranth (\emph{Amaranthus} \emph{Palmeri})
populations using genotyping-by-sequencing (GBS), \emph{Front Plant Sci}
\textbf{9} (2018).

\leavevmode\hypertarget{ref-farmer_evaluating_2017}{}%
18 Farmer JA, Webb EB, Pierce RA, and Bradley KW, Evaluating the
potential for weed seed dispersal based on waterfowl consumption and
seed viability, \emph{Pest Management Science} \textbf{73}:2592--2603
(2017).

\leavevmode\hypertarget{ref-norsworthy_-field_2014}{}%
19 Norsworthy JK, Griffith G, Griffin T, Bagavathiannan M, and Gbur EE,
In-field movement of glyphosate-resistant Palmer amaranth
(\emph{Amaranthus} \emph{Palmeri}) and its impact on cotton lint Yield:
Evidence supporting a zero-threshold strategy, \emph{Weed Science}
\textbf{62}:237--249 (2014).

\leavevmode\hypertarget{ref-schwartz-lazaro_resistance_2017}{}%
20 Schwartz-Lazaro LM, Norsworthy JK, Scott RC, and Barber LT,
Resistance of two Arkansas Palmer amaranth populations to multiple
herbicide sites of action, \emph{Crop Protection} \textbf{96}:158--163
(2017).

\leavevmode\hypertarget{ref-kumar_confirmation_2019}{}%
21 Kumar V, Liu R, Boyer G, and Stahlman PW, Confirmation of 2,4-D
resistance and identification of multiple resistance in a Kansas Palmer
amaranth (\emph{Amaranthus} \emph{Palmeri}) population, \emph{Pest
Management Science} \textbf{0} (2019).

\leavevmode\hypertarget{ref-chahal_glyphosate-resistant_2017}{}%
22 Chahal PS, Varanasi VK, Jugulam M, and Jhala AJ, Glyphosate-resistant
Palmer amaranth (\emph{Amaranthus} \emph{Palmeri}) in Nebraska:
Confirmation, EPSPS gene amplification, and response to POST corn and
soybean herbicides, \emph{Weed Technology} \textbf{31}:80--93 (2017).

\leavevmode\hypertarget{ref-gaines_molecular_2019}{}%
23 Gaines TA, Patterson EL, and Neve P, Molecular mechanisms of adaptive
evolution revealed by global selection for glyphosate resistance,
\emph{New Phytologist} \textbf{0} (2019).

\leavevmode\hypertarget{ref-gaines_mechanism_2011}{}%
24 Gaines TA, Shaner DL, Ward SM, Leach JE, Preston C, and Westra P,
Mechanism of Resistance of Evolved Glyphosate-Resistant Palmer Amaranth
(\emph{Amaranthus} \emph{Palmeri}), \emph{J Agric Food Chem}
\textbf{59}:5886--5889 (2011).

\leavevmode\hypertarget{ref-salas_resistance_2016}{}%
25 Salas RA, Burgos NR, Tranel PJ, Singh S, Glasgow L, and Scott RC
\emph{et al.}, Resistance to PPO-inhibiting herbicide in Palmer amaranth
from Arkansas, \emph{Pest Management Science} \textbf{72}:864--869
(2016).

\leavevmode\hypertarget{ref-salas-perez_frequency_2017}{}%
26 Salas-Perez RA, Burgos NR, Rangani G, Singh S, Refatti JP, and Piveta
L \emph{et al.}, Frequency of gly-210 deletion mutation among
protoporphyrinogen oxidase inhibitor--resistant Palmer amaranth
(\emph{Amaranthus} \emph{Palmeri}) populations, \emph{Weed Science}
\textbf{65}:718--731 (2017).

\leavevmode\hypertarget{ref-giacomini_two_2017-1}{}%
27 Giacomini DA, Umphres AM, Nie H, Mueller TC, Steckel LE, and Young BG
\emph{et al.}, Two new PPX2 mutations associated with resistance to
PPO-inhibiting herbicides in \emph{Amaranthus} \emph{Palmeri},
\emph{Pest Management Science} \textbf{73}:1559--1563 (2017).

\leavevmode\hypertarget{ref-rangani_novel_2019}{}%
28 Rangani G, Salas-Perez RA, Aponte RA, Knapp M, Craig IR, and Mietzner
T \emph{et al.}, A Novel Single-Site Mutation in the Catalytic Domain of
Protoporphyrinogen Oxidase IX (PPO) Confers Resistance to PPO-Inhibiting
Herbicides, \emph{Front Plant Sci} \textbf{10} (2019).

\leavevmode\hypertarget{ref-dominguez-valenzuela_first_2017}{}%
29 Dominguez-Valenzuela JA, Gherekhloo J, Fernández-Moreno PT,
Cruz-Hipolito HE, Alcántara-de la Cruz R, and Sánchez-González E
\emph{et al.}, First confirmation and characterization of target and
non-target site resistance to glyphosate in Palmer amaranth
(\emph{Amaranthus} \emph{Palmeri}) from Mexico, \emph{Plant Physiology
and Biochemistry} \textbf{115}:212--218 (2017).

\leavevmode\hypertarget{ref-varanasi_confirmation_2018}{}%
30 Varanasi VK, Brabham C, and Norsworthy JK, Confirmation and
Characterization of Non--target site Resistance to Fomesafen in Palmer
amaranth (Amaranthus palmeri), \emph{Weed Science} \textbf{66}:702--709
(2018).

\leavevmode\hypertarget{ref-doyle_rapid_1987}{}%
31 Doyle JJ and Doyle JL, A rapid DNA isolation procedure for small
quantities of fresh leaf tissue, \emph{Phytochemical Bull}
\textbf{19}:11--15 (1987).

\leavevmode\hypertarget{ref-wuerffel_distribution_2015}{}%
32 Wuerffel RJ, Young JM, Lee RM, Tranel PJ, Lightfoot DA, and Young BG,
Distribution of the G210 protoporphyrinogen oxidase mutation in Illinois
waterhemp (\emph{Amaranthus} \emph{Tuberculatus}) and an improved
molecular method for detection, \emph{Weed Science} \textbf{63}:839--845
(2015).

\leavevmode\hypertarget{ref-varanasi_statewide_2018}{}%
33 Varanasi VK, Brabham C, Norsworthy JK, Nie H, Young BG, and Houston M
\emph{et al.}, A Statewide survey of PPO-inhibitor resistance and the
prevalent target-site mechanisms in Palmer amaranth (\emph{Amaranthus}
\emph{Palmeri}) accessions from Arkansas, \emph{Weed Science}
\textbf{66}:149--158 (2018).

\leavevmode\hypertarget{ref-chatham_multistate_2015}{}%
34 Chatham LA, Bradley KW, Kruger GR, Martin JR, Owen MDK, and Peterson
DE \emph{et al.}, A Multistate study of the association between
glyphosate resistance and EPSPS gene amplification in waterhemp
(\emph{Amaranthus} \emph{Tuberculatus}), \emph{Weed Science}
\textbf{63}:569--577 (2015).

\leavevmode\hypertarget{ref-r_core_team_r:_2019}{}%
35 R: A Language and Environment for Statistical Computing, R Foundation
for Statistical Computing, Vienna, Austria (2019).

\leavevmode\hypertarget{ref-breiman_random_2001}{}%
36 Breiman L, Random Forests, \emph{Machine Learning} \textbf{45}:5--32
(2001).

\leavevmode\hypertarget{ref-biau_random_2016}{}%
37 Biau G and Scornet E, A random forest guided tour, \emph{TEST}
\textbf{25}:197--227 (2016).

\leavevmode\hypertarget{ref-langemeier_factors_2016}{}%
38 Langemeier CB, Robertson AE, Wang D, Jackson-Ziems TA, and Kruger GR,
Factors affecting the development and severity of goss's bacterial wilt
and leaf blight of corn, caused by \emph{Clavibacter}
\emph{Michiganensis} subsp. \emph{Nebraskensis}, \emph{Plant Disease}
\textbf{101}:54--61 (2016).

\leavevmode\hypertarget{ref-vieira_distribution_2018}{}%
39 Vieira BC, Samuelson SL, Alves GS, Gaines TA, Werle R, and Kruger GR,
Distribution of glyphosate-resistant \emph{Amaranthus} spp. In Nebraska,
\emph{Pest Management Science} \textbf{74}:2316--2324 (2018).

\leavevmode\hypertarget{ref-gaines_gene_2010}{}%
40 Gaines TA, Zhang W, Wang D, Bukun B, Chisholm ST, and Shaner DL
\emph{et al.}, Gene amplification confers glyphosate resistance in
\emph{Amaranthus} \emph{Palmeri}, \emph{Proceedings of the National
Academy of Sciences} \textbf{107}:1029--1034 (2010).

\leavevmode\hypertarget{ref-sammons_glyphosate_2014}{}%
41 Sammons RD and Gaines TA, Glyphosate resistance: State of knowledge,
\emph{Pest Management Science} \textbf{70}:1367--1377 (2014).

\leavevmode\hypertarget{ref-patterson_glyphosate_2018}{}%
42 Patterson EL, Pettinga DJ, Ravet K, Neve P, and Gaines TA, Glyphosate
resistance and EPSPS gene duplication: Convergent evolution in multiple
plant species, \emph{J Hered} \textbf{109}:117--125 (2018).

\leavevmode\hypertarget{ref-remnant_gene_2013}{}%
43 Remnant EJ, Good RT, Schmidt JM, Lumb C, Robin C, and Daborn PJ
\emph{et al.}, Gene duplication in the major insecticide target site,
Rdl, in \emph{Drosophila} \emph{Melanogaster}, \emph{Proc Natl Acad Sci
USA} \textbf{110}:14705--14710 (2013).

\leavevmode\hypertarget{ref-bass_gene_2011}{}%
44 Bass C and Field LM, Gene amplification and insecticide resistance,
\emph{Pest Management Science} \textbf{67}:886--890 (2011).

\leavevmode\hypertarget{ref-jugulam_tandem_2014}{}%
45 Jugulam M, Niehues K, Godar AS, Koo D-H, Danilova T, and Friebe B
\emph{et al.}, Tandem amplification of a chromosomal segment harboring
5-enolpyruvylshikimate-3-phosphate synthase locus confers glyphosate
resistance in \emph{Kochia} \emph{Scoparia}, \emph{Plant Physiology}
\textbf{166}:1200--1207 (2014).

\leavevmode\hypertarget{ref-giacomini_variable_2019}{}%
46 Giacomini DA, Westra P, and Ward SM, Variable Inheritance of
Amplified EPSPS Gene Copies in Glyphosate-Resistant Palmer Amaranth
(\emph{Amaranthus} \emph{Palmeri}), \emph{Weed Science}
\textbf{67}:176--182 (2019).

\leavevmode\hypertarget{ref-malone_epsps_2016}{}%
47 Malone JM, Morran S, Shirley N, Boutsalis P, and Preston C, EPSPS
gene amplification in glyphosate-resistant \emph{Bromus}
\emph{Diandrus}, \emph{Pest Management Science} \textbf{72}:81--88
(2016).

\leavevmode\hypertarget{ref-wiersma_gene_2015}{}%
48 Wiersma AT, Gaines TA, Preston C, Hamilton JP, Giacomini D, and Robin
Buell C \emph{et al.}, Gene amplification of
5-enol-pyruvylshikimate-3-phosphate synthase in glyphosate-resistant
\emph{Kochia} \emph{Scoparia}, \emph{Planta} \textbf{241}:463--474
(2015).

\leavevmode\hypertarget{ref-kumar_molecular_2015}{}%
49 Kumar V, Jha P, Giacomini D, Westra EP, and Westra P, Molecular basis
of evolved resistance to glyphosate and acetolactate synthase-inhibitor
herbicides in Kochia (\emph{Kochia} \emph{Scoparia}) accessions from
Montana, \emph{Weed sci} \textbf{63}:758--769 (2015).

\leavevmode\hypertarget{ref-koo_extrachromosomal_2018}{}%
50 Koo D-H, Molin WT, Saski CA, Jiang J, Putta K, and Jugulam M \emph{et
al.}, Extrachromosomal circular DNA-based amplification and transmission
of herbicide resistance in crop weed \emph{Amaranthus} \emph{Palmeri},
\emph{Proc Natl Acad Sci USA} \textbf{115}:3332--3337 (2018).

\leavevmode\hypertarget{ref-schafer_rhizosphere_2014}{}%
51 Schafer JR, Hallett SG, and Johnson WG, Rhizosphere Microbial
Community Dynamics in Glyphosate-Treated Susceptible and Resistant
Biotypes of Giant Ragweed (Ambrosia trifida), \emph{Weed Science}
\textbf{62}:370--381 (2014).

\leavevmode\hypertarget{ref-coburn_influence_2017}{}%
52 Coburn C, Influence of Experimental Methods on Herbicide Resistance
Confirmation, University of WyomingPh.D dissertation (2017).

\leavevmode\hypertarget{ref-jhala_confirmation_2014}{}%
53 Jhala AJ, Sandell LD, Rana N, Kruger GR, and Knezevic SZ,
Confirmation and control of triazine and 4-hydroxyphenylpyruvate
dioxygenase-inhibiting herbicide-resistant Palmer amaranth
(\emph{Amaranthus} \emph{Palmeri}) in Nebraska, \emph{Weed Technology}
\textbf{28}:28--38 (2014).

\leavevmode\hypertarget{ref-andrade_weed_2017}{}%
54 Andrade JF, Satorre EH, Ermácora CM, and Poggio SL, Weed communities
respond to changes in the diversity of crop sequence composition and
double cropping, \emph{Weed Research} \textbf{57}:148--158 (2017).

\leavevmode\hypertarget{ref-evans_managing_2016}{}%
55 Evans JA, Tranel PJ, Hager AG, Schutte B, Wu C, and Chatham LA
\emph{et al.}, Managing the evolution of herbicide resistance: Managing
the evolution of herbicide resistance, \emph{Pest Manag Sci}
\textbf{72}:74--80 (2016).

\leavevmode\hypertarget{ref-beckie_selecting_2009}{}%
56 Beckie HJ and Reboud X, Selecting for weed resistance: Herbicide
rotation and mixture, \emph{Weed Technology} \textbf{23}:363--370
(2009).

\leavevmode\hypertarget{ref-strom_characterization_2019}{}%
57 Strom SA, Gonzini LC, Mitsdarfer C, Davis AS, Riechers DE, and Hager
AG, Characterization of multiple herbicide--resistant waterhemp
(\emph{Amaranthus} \emph{Tuberculatus}) populations from Illinois to
VLCFA-inhibiting herbicides, \emph{Weed Science} \textbf{67}:369--379
(2019).

\leavevmode\hypertarget{ref-shergill_investigations_2018}{}%
58 Shergill LS, Barlow BR, Bish MD, and Bradley KW, Investigations of
2,4-D and multiple herbicide resistance in a Missouri waterhemp
(\emph{Amaranthus} \emph{Tuberculatus}) population, \emph{Weed Science}
\textbf{66}:386--394 (2018).

\leavevmode\hypertarget{ref-tranel_target-site_2017}{}%
59 Tranel PJ, Wu C, and Sadeque A, Target-site resistances to ALS and
PPO inhibitors are linked in waterhemp (\emph{Amaranthus}
\emph{Tuberculatus}), \emph{Weed Science} \textbf{65}:4--8 (2017).

\end{document}
